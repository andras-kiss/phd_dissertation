\chapter{Appendix}
%\pagestyle{headings}
\section{Diffusion and SECM scan simulation}
\begin{lstlisting}
program diffusion

implicit none

integer :: x_size, y_size, z_size
integer :: res, height
integer :: time_res

! 0.000703 mm^2/s = 703um^2/s for resolution of 1umx1um and 1s
! = 0.00000703 cm^2/s for resolution of 1umx1um and 1s
! CRC Handbook of chemistry and physics 87th 2006-2007.pdf page 848
! 1/2 Zn2+ -ra vonatkozik.
real :: const=0.017575 !for resolution of 20umx20um and 0.1s
! real :: const=0.02812 !for resolution of 50umx50um and 0.1s

integer :: h,i,j,k,x,y,m,switch,cells
real, dimension(0:101,0:101,0:101) :: a, b
real, dimension(1:100,1:100,1:100) :: flux
integer, dimension(1:100,1:100,1:100) :: mask
real :: pi=3.1415926535897932384626433832795
real maximum
real e0, x_real, y_real
real, dimension(0:100,0:100) :: sim
integer direction, divisions
real alpha, circumference, r_real
integer r, n
direction=-1
maximum=0.00000000000000
height=50

a=0.
b=0.
mask=1

! T�mb felt�lt�se. Ha azt akarod, hogy egy cella ne 
! v�ltozzon (Dirichlet), akkor azon a cell�n mask=0.
! Fluxus t�mb: minden k�rben hozz�ad�dik a fluxusban 
! t�rolt �rt�k a c-t�mbh�z. T�rbeli felbont�st a 
! D (const) hat�rozza meg.
do i=1, 100
  do j=1, 100
    if ( ((i-50)**2+(j-50)**2) < 20**2 ) then
      !20 itt a source sugara = 400um
      flux(i,j,1)=0.1
    endif
  end do
end do

open(1,file='flux at 1um.txt')
do i=0, 100
  do j=0, 100
    write(1, *) i*20, j*20, flux(i,j,1)
  end do
end do
close(1)

open(1,file='flux at 1um_final res.txt')
do i=0, 100, 5
  do j=0, 100, 5
    write(1, *) i*20, j*20, flux(i,j,1)
  end do
end do
close(1)

! Dirichlet felt�tel
! front �s h�ts� lap
do k=1, 100
  do i=1, 100
    mask(i, 1, k)=0
    mask(i, 100, k)=0
  end do
end do
! jobb �s bal lap
do k=1, 100
  do j=1, 100
    mask(1, j, k)=0
    mask(100, j, k)=0
  end do
end do
! fed�lap
do j=1, 100
  do i=1, 100
    do i=1, 100
      mask(i, j, 100)=0
    end do
end do

! Sz�mol�s. Az id�beli felbont�st D (const) hat�rozza meg.
! b �s a t�mb�k v�ltakoznak, �gy nem kell az eg�sz t�mb�t m�solgatni.
b=a
switch=0
! x=i, y=j, z=k, h=time
do h=1, 500 ! M A I N  L O O P

  ! All real cells computed. Not cycled: borders, which are all zeros.
  do k=1, 100
    do j=1, 100
      do i=1, 100
        if (mask(i,j,k)==1) then
          cells=6
          if ((k==1) .or. (k==100)) then
            cells=cells-1
          endif
          if ((j==1) .or. (j==100)) then
            cells=cells-1
          endif
          if ((i==1) .or. (i==100)) then
            cells=cells-1
          endif
          b(i,j,k)=a(i,j,k)+const*(a(i,j+1,k)+a(i-1,j,k)+a(i+1,j,k) &
          +a(i,j-1,k)+a(i,j,k-1)+a(i,j,k+1)-cells*a(i,j,k))
        endif
      end do
    end do
  end do
        
  ! Fluxusok hozz�ad�sa.
  do j=1, 100
    do i=1, 100
      b(i,j,1)=b(i,j,1)+flux(i,j,1)
    end do
  end do  
        
  ! a �s b t�mb�k cser�je
  a=b
  print *,h

end do
        
do i=0, 100
  do j=0, 100
    if (a(i,j,height)>maximum) then
      maximum=a(i,j,height)
    endif
  end do
end do  
print *,maximum
do i=0, 100
  do j=0, 100
    a(i,j,height)=a(i,j,height)/maximum
  end do
end do  
                                        
! O U T P U T   
open(1,file='real_100um_fullres.txt')
do i=0, 100
  do j=0, 100
    write(1, *) i*20, j*20, a(i,j,height)
  end do
end do          
close(1)

open(1,file='real_100um_finalres.txt')
do i=0, 100, 5
  do j=0, 100, 5
    write(1, *) i*20, j*20, a(i,j,height)
  end do
end do          
close(1)

! SECM scanning simulation
! FAST COMB
sim=0
e0=a(0,0,height)
open(1,file='fast_comb.txt')
do y=0, 100, 5
  do x=0, 100, 5
    sim(x,y) = a(x,y,height) + (e0-a(x,y,height))*0.8
    e0=sim(x,y)
    write(1, *) x*20, y*20, sim(x,y)
  end do
  do x=100, 0, 5
    e0 = a(x,y,height) + (e0-a(x,y,height))*0.9
  end do
end do
close(1)

! MEANDER
sim=0
e0=a(0,0,height)
open(1,file='meander.txt')
do y=0, 100, 5
  direction=direction*(-1)
  if (direction==1) then
    do x=0, 100, 5
      sim(x,y) = a(x,y,height) + (e0-a(x,y,height))*0.8
      e0=sim(x,y)
      write(1, *) x*20, y*20, sim(x,y)
    end do
  else
    do x=100, 0, -5
      sim(x,y) = a(x,y,height) + (e0-a(x,y,height))*0.8
      e0=sim(x,y)
      write(1, *) x*20, y*20, sim(x,y)
    end do
  endif
end do
close(1)

! COMB
sim=0
e0=a(0,0,height)
open(1,file='comb.txt')
open(2,file='comb_pattern.txt')
do y=0, 100, 5
  do x=0, 100, 5
    sim(x,y) = a(x,y,height) + (e0-a(x,y,height))*0.8
    e0=sim(x,y)
    !write(1, *) x*20, y*20, sim(x,y)
    write(2, *) x*20, y*20, sim(x,y)
  end do
  do x=100, 0, -5
    sim(x,y) = (sim(x,y) + a(x,y,height) + (e0-a(x,y,height))*0.8)/2
    e0=a(x,y,height) + (e0-a(x,y,height))*0.8
    write(1, *) x*20, y*20, sim(x,y)
    write(2, *) x*20, y*20, sim(x,y)
  end do
end do
close(1)
close(2)

! WEB
x_real=0
y_real=0
e0=a(50,50,height)
open(1,file='spider_net.txt')
do r=0, 10
  do alpha=0, 2*pi-pi/10, 2*pi/10
    x_real=100*r*cos(alpha)
    y_real=100*r*sin(alpha)
    x=nint((x_real+1000)/20)
    y=nint((y_real+1000)/20)
    !sim(x,y) = a(x,y,height) + (e0-a(x,y,height))*0.8
    e0=a(x,y,height)+(e0-a(x,y,height))*0.8
    write(1, *) nint(x_real), nint(y_real), e0
    print *,x_real,y_real
  end do
end do
close(1)

! ARC
x_real=0
y_real=0
e0=a(50,50,height)
open(1,file='stonehenge.txt')
write(1, *) 0, 0, e0
do r=0, 10
  circumference=2*r*100*pi
  divisions=circumference/100
  do alpha=0, 2*pi-2*pi/divisions, 2*pi/divisions
    x_real=100*r*cos(alpha)
    y_real=100*r*sin(alpha)
    x=nint((x_real+1000)/20)
    y=nint((y_real+1000)/20)
    !sim(x,y) = a(x,y,height) + (e0-a(x,y,height))*0.8
    e0=a(x,y,height)+(e0-a(x,y,height))*0.8
    write(1, *) nint(x_real), nint(y_real), e0
    print *,x_real,y_real
  end do
end do
close(1)
       
end program diffusion
\end{lstlisting}
	
	\section{Deconvolution program in Fortran}
\begin{lstlisting}
program deconvolution
implicit none

integer :: i, j, stat
real rc, e0, conv
real t

rc=0.85
open(1,file='data.txt')
open(2,file='data_deconvoluted.txt')
read(1, *) i, j, e0
do
   read(1, *, iostat=stat) i, j, conv
   if (stat /= 0) exit
   write(2, *) i, j, ((conv - e0*rc)/(1-rc))
   e0=conv
end do
close(1) 
close(2)

end program deconvolution
\end{lstlisting}
