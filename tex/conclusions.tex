\chapter{Conclusions}
%\addcontentsline{toc}{chapter}{Conclusions}
\pagestyle{plain}
The present work has been devoted to improve potentiometric Scanning Electrochemical Microscopy.
Scanning is relatively slow due to the long response time of the potentiometric measuring cell.
Shortened scanning time is useful when the studied system is changing.
When scanned too fast however, distortion is added to the image.
I've successfully sped up the technique without compromising image quality.
In another effort, I've managed to separate the effect of electric field from the Nernstian potential response of the ion selective microelectrode. 

The main results are summarized in the thesis points:

\begin{enumerate}
\item I have prepared solid-contact magnesium ion-selective micropipettes for the first time.
I have compared them to conventional, liquid contact microelectrodes by basic characterization and model system study to prove the improved performance.
I have shown the improved quality of potentiometric SECM images recorded with them.

\item Taking advantage of the new solid-contact microelectrodes, I have studied the galvanic corrosion of magnesium and the AZ63 magnesium alloy in corrosive electrolyte by mapping the concentration distribution of dissolving magnesium ions.
The use of the new solid-contact ion-selective microelectrodes resulted less distorted images.

\item I have used a novel experimental method to measure the local flux of magnesium ions over the corroding surface, and the results agreed very well with the direct measurement of corrosion current.
After applying Faraday's Law of Electrolysis, the two results could be compared.
This shows the applicability of the new solid-contact magnesium ion-selective microelectrodes in obtaining quantitative results.

\newpage
\item I have designed new scanning patterns and algorithms, optimized to radially symmetric targets.
I've proven that with these new patterns and algorithms, image distortion is lower compared to the conventional ones, by numerical simulations and experimental SECM scans.

\item I was the first to use deconvolution to reduce distortion in potentiometric SECM images.
I have shown that distortion caused by the large time constant can be reduced by this technique.
To prove the benefits of the technique, I have compared deconvoluted linescans to the real concentration disrtibution.
I have shown the improved quality of deconvoluted potentiometric SECM images.

\item I have successfully used deconvolution to restore a potentiometric SECM image about a corroding carbon steel sample.
Evaluation of this data was possible, because scanning time \emph{and} distortion was reduced at the same time.

\item I've shown the applicability of blind deconvolution.
This method can be used on measurements where the parameters of the deconvolution function are unknown.

\item I have successfully resolved the observed discrepancy in recent papers featuring highly overestimated apparent ion activities in potentiometric SECM images.
I have explained this effect by the electric field present in many studied systems -- galvanically corroding ones in particular -- that has a direct influence on the measured potential.
I have shown how big of an error can it cause.
In the system I have studied, the error would have been almost four orders of magnitude.
By taking this effect into account, a more accurate conclusion could be drawn.

\end{enumerate}
