\chapter*{Motivation}
\addcontentsline{toc}{chapter}{Motivation}
I started my doctoral studies at the Department of General and Physical Chemistry at the University of Pécs in 2011 under the supervision of professor Géza Nagy.
By that time, I've already spent 3 years there as an undergraduate student, culminating in my MSc thesis, which I successfully defended in 2011.
The original title of my PhD research was \emph{,,Application of potentiometric microelectrodes in Scanning Electrochemical Microscopy to study the corrosion of magnesium and its alloys''}.
I started the work with great enthusiasm, but soon discovered, that the potentiometric SECM had one major limitation which prevented me from pursuing my studies in corrosion science.
The method was too slow to complete a relevant portion of the scan before the system has completely changed.
Corrosion is highly localized, and the location and size of the anodic and cathodic spots are quick to change.
On the other hand, when one increases scanning speed, the image becomes distorted.
This is because the time allowed for the potentiometric cell to reach equilibrium potential is getting closer to $\tau$, the time constant of the cell, or more specifically 4$\tau$, which is the least amount of time necessary to reach equilibrium potential to a reasonable extent.
In order to use SECM more effectively in corrosion studies, one has to speed up the method, \emph{and} lower, or at least not increase imaging distortion.

Efforts have already been made at the Department of General and Physical Chemistry to lower the resistance of the electrodes used as SECM measuring probes.
By using novel solid contact instead of the conventional liquid contact as interface between the metallic conductor and the ion selective cocktail, resistance, and therefore the RC time constant could be decreased by a factor of ten.
Building on the new electrodes, several papers have been published about successful research collaborations with neurophysiologists, botanists, and in particular, corrosion scientists.

I joined the research with the hope of a worthwhile contribution to the technique of potentiometric SECM, which eventually might lead to new possibilities in corrosion science and other areas.
My PhD research was slowly turning into a methodological study, exploring ideas to improve the technique.

In 2012 I was lucky enough to be able to participate in the biggest conference of ,,SECM and related techniques'', which in that year was held in Ein Gedi, Israel.
I've realized, that other research groups are also struggling with the difficulties of potentiometric SECM, and the problem is not limited to corrosion science.
In fact, most of the studies are done in amperometric mode, which, based on discussions with participants, might be exactly due to these difficulties.
Also in 2012 I was able to spend a month at the Department of Physical Chemistry of the La Laguna University in Tenerife under the supervision of professor Ricardo M. Souto.
We used the very same solid contact magnesium-ion selective electrodes I prepared in Pécs to map magnesium-ion concentration above corroding magnesium and magnesium alloy samples.
They worked unexpectedly well, and we were able to take high-speed, low distortion images of the samples for the first time.
We published the results.
I was very happy with the two papers done in cooperation between our research groups, and the methodological aspect of them fitted nicely in my research topic.

When I returned home, I immediately started working on a new idea I had when I was in Tenerife.
Since the majority of the targets in SECM studies are, or can be made circular, it makes sense to use a scanning pattern based on the polar coordinate system instead of the conventional 2D raster based on the Cartesian coordinate system.
With new scanning patterns and algorithms, I managed to further decrease distortion and increase scanning speed at the same time.

While working on the final touches of the paper about the new scanning algorithms, I had another idea.
The transient response of the potentiometric cell due to concentration changes is described by a relatively simple function.
It's not necessary to wait 4$\tau$ to find out the equilibrium potential at a given sampling point.
From the initial potential, and the potential at time $t$, it's possible to calculate it.
One has to know the response characteristics of the cell, but that can be easily measured.
I managed to increase scanning speed even further.
My PhD thesis is mainly about these three improvements to the potentiometric SECM.

In May 2016, I was lucky enough to be invited to the Analytica 2016 conference in Münich, by professor Frank-Michael Matysik.
I presented my work, and got precious feedback from the audience, including two of the most prominent researches working in the field; professor Michael Mirkin and professor Eric Bakker.
The discussions with them shed some light on a few issues I faced writing my thesis.

The most recent topic of my dissertation is about resolving a discrepancy that I encountered in 2011 when I started studying the galvanic corrosion of magnesium.
The ion-selective microelectrode reported impossibly high magnesium ion activities.
I suspected a contribution from the electric field that is the consequence of the potential difference between the surfaces of the metals constituting the galvanic couple.
With a series of experiments I managed to prove that it was indeed the case, and I have shown how big of an error can it cause. 

The author hopes that he managed to contribute to the field of Scanning Electrochemical Microscopy with this thesis.
Please forgive the rather lengthy discussion about the role of biologists in the development of microelectrodes, but the author himself graduated as a biologists, and takes pride in the achievements of the early pioneers of electroneurophysiology.
