\chapter*{Preface}
\addcontentsline{toc}{chapter}{Preface}
The work presented here was performed mainly at the \emph{Department of General and Physical Chemistry} in the \emph{Doctoral School of Chemistry} at the \emph{University of Pécs}, during the years 2011-2016, under the supervision of Professor Géza Nagy.
Some of the work was done at the \emph{Department of Physical Chemistry} of the \emph{La Laguna University} in Tenerife, Spain, under the joint supervision of Professor Géza Nagy, and Professor Ricardo M. Souto.
This thesis is based almost entirely on the following publications, which are referred to in the text by their Roman numerals.

\begin{enumerate}[{\color{blue} \bf I.}]
\item \textbf{András Kiss}, Ricardo M. Souto, Géza Nagy

Investigation of Mg/Al alloy sacrificial anode corrosion with Scanning Electrochemical Microscopy

\emph{Periodica Polytechnica Chemical Engineering 57, no. 1-2 (2013): 11-14.}

IF.: 0.30, cited by: 2

\item Javier Izquierdo, \textbf{András Kiss}, Juan José Santana, Lívia Nagy, István Bitter, Hugh S. Isaacs, Géza Nagy, Ricardo M. Souto

Development of Mg$^{2+}$ ion-selective microelectrodes for potentiometric scanning electrochemical microscopy monitoring of galvanic corrosion processes

\emph{Journal of The Electrochemical Society 160, no. 9 (2013): C451-C459.}

IF.: 3.27, cited by: 14

\item Ricardo M. Souto, \textbf{András Kiss}, Javier Izquierdo, Lívia Nagy, István Bitter, Géza Nagy

Spatially-resolved imaging of concentration distributions on corroding mag\-ne\-si\-um-based materials exposed to aqueous environments by SECM

\emph{Electrochemistry Communications 26 (2013): 25-28.}

IF.: 4.85, cited by: 20

\item \textbf{András Kiss}, Géza Nagy

New SECM scanning algorithms for improved potentiometric imaging of circularly symmetric targets 

\emph{Electrochimica Acta 119 (2014): 169-174.}

IF.: 4.50, cited by: 5

\item \textbf{András Kiss}, Géza Nagy

Deconvolution in Potentiometric SECM

\emph{Electroanalysis 27, no. 3 (2015): 587-590.}

IF.: 2.14, cited by: 1

\item \textbf{András Kiss}, Géza Nagy

Deconvolution of potentiometric SECM images recorded with high scan rate

\emph{Electrochimica Acta 163 (2015): 303-309.}

IF.: 4.50, cited by: 3

\end{enumerate}
\section*{Contribution statement}
Publications \textbf{I}, and \textbf{IV-VI} are entirely my own work including the original idea, the experimental work, the simulations, the calculations, the written text and the figures. They were done under the guidance of my doctoral supervisor, Professor Géza Nagy, in Pécs, at the Department of General and Physical Chemistry of the University of Pécs.
Publications \textbf{II} and \textbf{III} are mostly my work.
The work published therein was done at the University of La Laguna, Tenerife, under the supervision of Professor Ricardo M. Souto and Professor Géza Nagy.
In \textbf{II} and \textbf{III}, I prepared the electrodes, did the electrode characterization and the majority of the SECM scans. 
\section*{Not included in the thesis}

\begin{enumerate}[(a)]
\item Zsuzsanna \H{O}ri, \textbf{András Kiss}, Anton Alexandru Ciucu, Constantin Mihailciuc, Cristian Dragos Stefanescu, Livia Nagy, Géza Nagy

Sensitivity enhancement of a ,,bananatrode'' biosensor for dopamine based on SECM studies inside its reaction layer

IF.: 4.10, cited by: 4

\emph{Sensors and Actuators B: Chemical 190 (2014): 149-156.}

\item Javier Izquierdo, Bibiana M Fernández-Pérez, Dániel Filotás, Zsuzsanna Őri, \textbf{András Kiss}, Romen T Martín-Gómez, Lívia Nagy, Géza Nagy, Ricardo M Souto

Imaging of Concentration Distributions and Hydrogen Evolution on Corroding Magnesium Exposed to Aqueous Environments Using Scanning Electrochemical Microscopy

IF.: 2.471

\emph{Electroanalysis (2016).}

\item A. El Jaouhari,  Dániel Filotás, \textbf{András Kiss}, M. Laabd, E. A. Bazzaoui, Lívia Nagy, Géza Nagy, A. Albourine, J. I. Martins, R. Wang

SECM investigation of electrochemically synthesized polypyrrole from aqueous medium

IF.: 2.223

\emph{Journal of Applied Electrochemistry, (2016).}

\end{enumerate}

\section*{Published prior PhD}
\begin{itemize}
\item[] \textbf{András Kiss}, László Kiss, Barna Kovács, Géza Nagy

Air Gap Microcell for Scanning Electrochemical Microscopic Imaging of Carbon Dioxide Output. Model Calculation and Gas Phase SECM Measurements for Estimation of Carbon Dioxide Producing Activity of Microbial Sources

\emph{Electroanalysis 23, no. 10 (2011): 2320-2326.}

IF.: 2.14, cited by: 1

\end{itemize}

\section*{Published in non-refereed journal}
\begin{itemize}

\item[] Lívia Nagy, Gergely Gyetvai, \textbf{András Kiss}, Ricardo Souto, Javier Izquierdo, Géza Nagy

Speciális célra szolgáló mikroelektródok kifejlesztése és alkalmazása

\emph{Magyar Kémiai Folyóirat 119, 2-3. (2013): 104-109.}
\end{itemize}

%\newpage
\section*{Book chapter}
\begin{itemize}

\item[] Ricardo M. Souto, Javier Izquierdo, Juan J. Santana, \textbf{András Kiss}, Lívia Nagy, Géza Nagy

Progress in Scanning Electrochemical Microscopy by coupling Potentiometric and Amperometric Measurement Modes

\emph{Current Microscopy Contributions to Advances in Science and Technology (A. Méndez-Vilas, Ed.), Formatex, (2012): 1407-1415.}
\end{itemize}


\section*{Presentations}

\section*{Posters}
